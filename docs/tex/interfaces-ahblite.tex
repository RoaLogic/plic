\subsection{AHB-Lite Interface}

The AHB-Lite interface is a regular AHB-Lite slave port. All signals are
supported. See the
\emph{\href{https://www.arm.com/products/system-ip/amba-specifications}{AMBA
3 AHB-Lite Specification}} for a complete description of the signals.

\begin{longtable}[]{@{}lccl@{}}
\toprule
\textbf{Port} & \textbf{Size} & \textbf{Direction} & \textbf{Description}\tabularnewline
\midrule
\endhead
\texttt{HRESETn} & 1 & Input & Asynchronous active low reset\tabularnewline
\texttt{HCLK} & 1 & Input & Clock Input\tabularnewline
\texttt{HSEL} & 1 & Input & Bus Select\tabularnewline
\texttt{HTRANS} & 2 & Input & Transfer Type\tabularnewline
\texttt{HADDR} & \texttt{HADDR\_SIZE} & Input & Address Bus\tabularnewline
\texttt{HWDATA} & \texttt{HDATA\_SIZE} & Input & Write Data Bus\tabularnewline
\texttt{HRDATA} & \texttt{HDATA\_SIZE} & Output & Read Data Bus\tabularnewline
\texttt{HWRITE} & 1 & Input & Write Select\tabularnewline
\texttt{HSIZE} & 3 & Input & Transfer Size\tabularnewline
\texttt{HBURST} & 3 & Input & Transfer Burst Size\tabularnewline
\texttt{HPROT} & 4 & Input & Transfer Protection Level\tabularnewline
\texttt{HREADYOUT} & 1 & Output & Transfer Ready Output\tabularnewline
\texttt{HREADY} & 1 & Input & Transfer Ready Input\tabularnewline
\texttt{HRESP} & 1 & Output & Transfer Response\tabularnewline
\bottomrule
\caption{AHB Interface Signals}
\label{tab:AHBIF}
\end{longtable}

\subsubsection{HRESETn}

When the active low asynchronous \texttt{HRESETn} input is asserted
(`0'), the interface is put into its initial reset state.

\subsubsection{HCLK}

\texttt{HCLK} is the interface system clock. All internal logic for the
AMB3-Lite interface operates at the rising edge of this system clock and
AHB bus timings are related to the rising edge of \texttt{HCLK}.

\subsubsection{HSEL}

The AHB-Lite interface only responds to other signals on its bus -- with
the exception of the global asynchronous reset signal \texttt{HRESETn}
-- when \texttt{HSEL} is asserted (`1'). When \texttt{HSEL} is negated
(`0') the interface considers the bus \texttt{IDLE}.

\subsubsection{HTRANS}

HTRANS indicates the type of the current transfer.

\begin{longtable}[]{@{}ccp{7cm}@{}}
\toprule
\textbf{HTRANS} & \textbf{Type} & \textbf{Description}\tabularnewline
\midrule
\endhead
00 & \texttt{IDLE} & No transfer required\tabularnewline
01 & \texttt{BUSY} & Connected master is not ready to accept data, but intents to continue the current burst.\tabularnewline
10 & \texttt{NONSEQ} & First transfer of a burst or a single transfer\tabularnewline
11 & \texttt{SEQ} & Remaining transfers of a burst\tabularnewline
\bottomrule
\caption{HTRANS Transfer Types}
\label{table:HTRANS}
\end{longtable}

\subsubsection{HADDR}

\texttt{HADDR} is the address bus. Its size is determined by the
\texttt{HADDR\_SIZE} parameter and is driven to the connected
peripheral.

\subsubsection{HWDATA}

\texttt{HWDATA} is the write data bus. Its size is determined by the
\texttt{HDATA\_SIZE} parameter and is driven to the connected
peripheral.

\subsubsection{HRDATA}

\texttt{HRDATA} is the read data bus. Its size is determined by
\texttt{HDATA\_SIZE} parameter and is sourced by the APB4 peripheral.

\subsubsection{HWRITE}

\texttt{HWRITE} is the read/write signal. \texttt{HWRITE} asserted (`1')
indicates a write transfer.

\subsubsection{HSIZE}

\texttt{HSIZE} indicates the size of the current transfer.

\begin{longtable}[]{@{}ccl@{}}
\toprule
\textbf{HSIZE} & \textbf{Size} & \textbf{Description}\tabularnewline
\midrule
\endhead
000 & 8 bit & Byte\tabularnewline
001 & 16 bit & Half Word\tabularnewline
010 & 32 bit & Word\tabularnewline
011 & 64 bits & Double Word\tabularnewline
100 & 128 bit &\tabularnewline
101 & 256 bit &\tabularnewline
110 & 512 bit &\tabularnewline
111 & 1024 bit &\tabularnewline
\bottomrule
\caption{HSIZE Values}
\label{tab:HSIZE}
\end{longtable}

\subsubsection{HBURST}

HBURST indicates the transaction burst type -- a single transfer or part
of a burst.

\begin{longtable}[]{@{}ccl@{}}
\toprule
\textbf{HBURST} & \textbf{Type} & \textbf{Description}\tabularnewline
\midrule
\endhead
000 & \texttt{SINGLE} & Single access**\tabularnewline
001 & \texttt{INCR} & Continuous incremental burst\tabularnewline
010 & \texttt{WRAP4} & 4-beat wrapping burst\tabularnewline
011 & \texttt{INCR4} & 4-beat incrementing burst\tabularnewline
100 & \texttt{WRAP8} & 8-beat wrapping burst\tabularnewline
101 & \texttt{INCR8} & 8-beat incrementing burst\tabularnewline
110 & \texttt{WRAP16} & 16-beat wrapping burst\tabularnewline
111 & \texttt{INCR16} & 16-beat incrementing burst\tabularnewline
\bottomrule
\caption{HBURST Types}
\label{tab:HBURST}
\end{longtable}

\subsubsection{HPROT}

The \texttt{HPROT} signals provide additional information about the bus
transfer and are intended to implement a level of protection.

\begin{longtable}[]{@{}ccl@{}}
\toprule
\textbf{Bit\#} & \textbf{Value} & \textbf{Description}\tabularnewline
\midrule
\endhead
3 & 1 & Cacheable region addressed\tabularnewline
& 0 & Non-cacheable region addressed\tabularnewline
2 & 1 & Bufferable\tabularnewline
& 0 & Non-bufferable\tabularnewline
1 & 1 & Privileged Access\tabularnewline
& 0 & User Access\tabularnewline
0 & 1 & Data Access\tabularnewline
& 0 & Opcode fetch\tabularnewline
\bottomrule
\caption{HPROT Indicators}
\label{tab:HPROT}
\end{longtable}

\subsubsection{HREADYOUT}

\texttt{HREADYOUT} indicates that the current transfer has finished.
Note, for the AHB-Lite PLIC this signal is constantly asserted as the
core is always ready for data access.

\subsubsection{HREADY}

\texttt{HREADY} indicates whether or not the addressed peripheral is
ready to transfer data. When \texttt{HREADY} is negated (`0') the
peripheral is not ready, forcing wait states. When \texttt{HREADY} is
asserted (`1') the peripheral is ready and the transfer completed.

\subsubsection{HRESP}

\texttt{HRESP} is the instruction transfer response and indicates OKAY
(`0') or ERROR (`1').
